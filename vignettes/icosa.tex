\documentclass[]{article}
\usepackage{lmodern}
\usepackage{amssymb,amsmath}
\usepackage{ifxetex,ifluatex}
\usepackage{fixltx2e} % provides \textsubscript
\ifnum 0\ifxetex 1\fi\ifluatex 1\fi=0 % if pdftex
  \usepackage[T1]{fontenc}
  \usepackage[utf8]{inputenc}
\else % if luatex or xelatex
  \ifxetex
    \usepackage{mathspec}
  \else
    \usepackage{fontspec}
  \fi
  \defaultfontfeatures{Ligatures=TeX,Scale=MatchLowercase}
\fi
% use upquote if available, for straight quotes in verbatim environments
\IfFileExists{upquote.sty}{\usepackage{upquote}}{}
% use microtype if available
\IfFileExists{microtype.sty}{%
\usepackage[]{microtype}
\UseMicrotypeSet[protrusion]{basicmath} % disable protrusion for tt fonts
}{}
\PassOptionsToPackage{hyphens}{url} % url is loaded by hyperref
\usepackage[unicode=true]{hyperref}
\hypersetup{
            pdftitle={Introduction to the R package icosa v0.10.0 for global triangular and hexagonal gridding},
            pdfauthor={Adam T. Kocsis},
            pdfborder={0 0 0},
            breaklinks=true}
\urlstyle{same}  % don't use monospace font for urls
\usepackage[margin=1in]{geometry}
\usepackage{color}
\usepackage{fancyvrb}
\newcommand{\VerbBar}{|}
\newcommand{\VERB}{\Verb[commandchars=\\\{\}]}
\DefineVerbatimEnvironment{Highlighting}{Verbatim}{commandchars=\\\{\}}
% Add ',fontsize=\small' for more characters per line
\usepackage{framed}
\definecolor{shadecolor}{RGB}{248,248,248}
\newenvironment{Shaded}{\begin{snugshade}}{\end{snugshade}}
\newcommand{\KeywordTok}[1]{\textcolor[rgb]{0.13,0.29,0.53}{\textbf{#1}}}
\newcommand{\DataTypeTok}[1]{\textcolor[rgb]{0.13,0.29,0.53}{#1}}
\newcommand{\DecValTok}[1]{\textcolor[rgb]{0.00,0.00,0.81}{#1}}
\newcommand{\BaseNTok}[1]{\textcolor[rgb]{0.00,0.00,0.81}{#1}}
\newcommand{\FloatTok}[1]{\textcolor[rgb]{0.00,0.00,0.81}{#1}}
\newcommand{\ConstantTok}[1]{\textcolor[rgb]{0.00,0.00,0.00}{#1}}
\newcommand{\CharTok}[1]{\textcolor[rgb]{0.31,0.60,0.02}{#1}}
\newcommand{\SpecialCharTok}[1]{\textcolor[rgb]{0.00,0.00,0.00}{#1}}
\newcommand{\StringTok}[1]{\textcolor[rgb]{0.31,0.60,0.02}{#1}}
\newcommand{\VerbatimStringTok}[1]{\textcolor[rgb]{0.31,0.60,0.02}{#1}}
\newcommand{\SpecialStringTok}[1]{\textcolor[rgb]{0.31,0.60,0.02}{#1}}
\newcommand{\ImportTok}[1]{#1}
\newcommand{\CommentTok}[1]{\textcolor[rgb]{0.56,0.35,0.01}{\textit{#1}}}
\newcommand{\DocumentationTok}[1]{\textcolor[rgb]{0.56,0.35,0.01}{\textbf{\textit{#1}}}}
\newcommand{\AnnotationTok}[1]{\textcolor[rgb]{0.56,0.35,0.01}{\textbf{\textit{#1}}}}
\newcommand{\CommentVarTok}[1]{\textcolor[rgb]{0.56,0.35,0.01}{\textbf{\textit{#1}}}}
\newcommand{\OtherTok}[1]{\textcolor[rgb]{0.56,0.35,0.01}{#1}}
\newcommand{\FunctionTok}[1]{\textcolor[rgb]{0.00,0.00,0.00}{#1}}
\newcommand{\VariableTok}[1]{\textcolor[rgb]{0.00,0.00,0.00}{#1}}
\newcommand{\ControlFlowTok}[1]{\textcolor[rgb]{0.13,0.29,0.53}{\textbf{#1}}}
\newcommand{\OperatorTok}[1]{\textcolor[rgb]{0.81,0.36,0.00}{\textbf{#1}}}
\newcommand{\BuiltInTok}[1]{#1}
\newcommand{\ExtensionTok}[1]{#1}
\newcommand{\PreprocessorTok}[1]{\textcolor[rgb]{0.56,0.35,0.01}{\textit{#1}}}
\newcommand{\AttributeTok}[1]{\textcolor[rgb]{0.77,0.63,0.00}{#1}}
\newcommand{\RegionMarkerTok}[1]{#1}
\newcommand{\InformationTok}[1]{\textcolor[rgb]{0.56,0.35,0.01}{\textbf{\textit{#1}}}}
\newcommand{\WarningTok}[1]{\textcolor[rgb]{0.56,0.35,0.01}{\textbf{\textit{#1}}}}
\newcommand{\AlertTok}[1]{\textcolor[rgb]{0.94,0.16,0.16}{#1}}
\newcommand{\ErrorTok}[1]{\textcolor[rgb]{0.64,0.00,0.00}{\textbf{#1}}}
\newcommand{\NormalTok}[1]{#1}
\usepackage{graphicx,grffile}
\makeatletter
\def\maxwidth{\ifdim\Gin@nat@width>\linewidth\linewidth\else\Gin@nat@width\fi}
\def\maxheight{\ifdim\Gin@nat@height>\textheight\textheight\else\Gin@nat@height\fi}
\makeatother
% Scale images if necessary, so that they will not overflow the page
% margins by default, and it is still possible to overwrite the defaults
% using explicit options in \includegraphics[width, height, ...]{}
\setkeys{Gin}{width=\maxwidth,height=\maxheight,keepaspectratio}
\IfFileExists{parskip.sty}{%
\usepackage{parskip}
}{% else
\setlength{\parindent}{0pt}
\setlength{\parskip}{6pt plus 2pt minus 1pt}
}
\setlength{\emergencystretch}{3em}  % prevent overfull lines
\providecommand{\tightlist}{%
  \setlength{\itemsep}{0pt}\setlength{\parskip}{0pt}}
\setcounter{secnumdepth}{0}
% Redefines (sub)paragraphs to behave more like sections
\ifx\paragraph\undefined\else
\let\oldparagraph\paragraph
\renewcommand{\paragraph}[1]{\oldparagraph{#1}\mbox{}}
\fi
\ifx\subparagraph\undefined\else
\let\oldsubparagraph\subparagraph
\renewcommand{\subparagraph}[1]{\oldsubparagraph{#1}\mbox{}}
\fi

% set default figure placement to htbp
\makeatletter
\def\fps@figure{htbp}
\makeatother

\usepackage{etoolbox}
\makeatletter
\providecommand{\subtitle}[1]{% add subtitle to \maketitle
  \apptocmd{\@title}{\par {\large #1 \par}}{}{}
}
\makeatother
% https://github.com/rstudio/rmarkdown/issues/337
\let\rmarkdownfootnote\footnote%
\def\footnote{\protect\rmarkdownfootnote}

% https://github.com/rstudio/rmarkdown/pull/252
\usepackage{titling}
\setlength{\droptitle}{-2em}

\pretitle{\vspace{\droptitle}\centering\huge}
\posttitle{\par}

\preauthor{\centering\large\emph}
\postauthor{\par}

\predate{\centering\large\emph}
\postdate{\par}

\title{Introduction to the R package `icosa' v0.10.0 for global triangular and
hexagonal gridding}
\author{Adam T. Kocsis}
\date{2020-02-14}

\begin{document}
\maketitle

\section{1. Introduction}\label{introduction}

The purpose of this vignette is to demonstrate the basic usage of the
\texttt{icosa} package, explain object structures and basic
functionalities. The primary targeted application of the package is in
global biological sciences (e.g.~in macroecological, biogeographical
analyses), but other fields might find the structures and procedures
relevant, given that they operate with point coordinate data. This is
just a brief introduction to the package's capabilities. The complete
documentation of the package, and relevant tutorials will be posten on
the evolv-ED blog (\url{http://www.evolv-ed.net}). Previous versions of
this document are available at the package's GitHub repository
(\url{https://github.com/adamkocsis/icosa/_archive/vignettes}).

\section{2. The grids}\label{the-grids}

The primary problem with ecological samples is that due to density and
uniformity issues, the data points are to be aggregated to distinct
units. As coordinate recording is very efficient on the 2d surface of a
polar coordinate system (i.e.~latiude and longitude data), this was
primarly achieved by rectangular gridding of the surface (for instance
1° times 1° grid cells). Unfortunately, this method suffers from
systematic biasing effects: as the poles are approached, the cells
become smaller in area, and come closer together.

The \texttt{icosa} package apptrroaches this problem from one of the
most straightforward ways, by tessellation of a regular icosahedron to a
given resolution. This procedure ends up with a polyhedral object of
triangular faces of higly isometric properties: very similar shapes of
cells which are roughly equally distanced, and similar in cell area.

To visualize the grids in 3 dimension, you have to have rgl package
installed. This is optional, but to ensure maximum functionality when
plotting with `rgl', it is better to attach the `rgl' package first,
otherwise it will mask out some functionalities of `icosa'. Attaching
the package `rgdal' is necessary for reading in shapefiles and
transformation of projections, while the raster package is required for
some lookup functions.

\begin{Shaded}
\begin{Highlighting}[]
\KeywordTok{library}\NormalTok{(rgdal)}
\KeywordTok{library}\NormalTok{(raster)}
\KeywordTok{library}\NormalTok{(rgl)}
\KeywordTok{library}\NormalTok{(icosa)}
\end{Highlighting}
\end{Shaded}

\subsection{2.1. Grid creation}\label{grid-creation}

To create a triangular grid use the function \texttt{trigrid()}:

\begin{Shaded}
\begin{Highlighting}[]
\CommentTok{# create a trigrid class object}
\NormalTok{tri <-}\StringTok{ }\KeywordTok{trigrid}\NormalTok{()}

\CommentTok{# the show() method displays basic information of the package}
\NormalTok{tri}
\end{Highlighting}
\end{Shaded}

\begin{verbatim}
## A/An trigrid object with 12 vertices, 30 edges and 20 faces.
## The mean grid edge length is 7053.65 km or 63.43 degrees.
## Use plot3d() to see a 3d render.
\end{verbatim}

\begin{Shaded}
\begin{Highlighting}[]
\CommentTok{# plot the object in 3d}
\KeywordTok{plot3d}\NormalTok{(tri, }\DataTypeTok{guides=}\NormalTok{F)}
\end{Highlighting}
\end{Shaded}

\includegraphics{icosa_files/figure-latex/first-1.png}

Without any specified additional entry, the first line will create an
icosahedron with the center of \texttt{c(0,0,0)} Cartesian coordinates
and the `R2' (authalic, as defined by IUGG (1)) radius of Earth between
the object center and the vertices. These can be altered by setting the
\texttt{radius} and \texttt{center} arguments if necessary. When dealing
with properly georeferenced data, the model ellipsoid (or in this case,
the sphere) is to be taken into account when the data and the grid
interact. Therefore a slot called \texttt{proj4strig} is added to the
grid object, which contains a CRS class string generated automatically
from the input radius. With the default settings this is:

\begin{Shaded}
\begin{Highlighting}[]
\NormalTok{tri}\OperatorTok{@}\NormalTok{proj4string}
\end{Highlighting}
\end{Shaded}

\begin{verbatim}
## CRS arguments: +proj=longlat +a=6371007 +b=6371007
\end{verbatim}

Setting the first argument of the \texttt{trigrid()} function will
create more complex objects that have tessellated faces:

\begin{Shaded}
\begin{Highlighting}[]
\CommentTok{# create a trigrid class object}
\NormalTok{gLow <-}\StringTok{ }\KeywordTok{trigrid}\NormalTok{(}\DataTypeTok{tessellation=}\KeywordTok{c}\NormalTok{(}\DecValTok{4}\NormalTok{,}\DecValTok{4}\NormalTok{))}

\CommentTok{# plot the object in 3d}
\KeywordTok{plot3d}\NormalTok{(gLow, }\DataTypeTok{guides=}\NormalTok{F)}
\end{Highlighting}
\end{Shaded}

\includegraphics{icosa_files/figure-latex/trigrid-1.png}

The result is another \texttt{trigrid} class object with the
tessellation vector of \texttt{c(4,4)}. The tessellation vector is the
primary argument influencing grid resolution. It consists of integer
values which are larger than 1. These values will be passed in sequence
to the tessellation function, using the result of the previous round as
an input. In the example of the \texttt{c(4,4)} grid, the icosahedron
will be tessellated with the value of 4 in the first round, meaning that
every edge of the 20 faces are split to 4, which then results in 4 times
4 new triangular faces instead of the one original (4 times 4 times 20
new faces in total). The second round will be repeated for every newly
formed face as well, so the total resolution of the grid will be 4 times
4 times 4 times 4 times 20 faces.

The obvious question is then: what is the difference between the
\texttt{c(2,2,2,2))}, \texttt{c(4,4)}, \texttt{c(8,2)}, \texttt{c(2,8)}
and \texttt{c(16)} grids? The answer depends on the applied tessellation
method. The icosahedron itself is smaller in surface area and volume
than the sphere. The points created between the faces need to be
projected to the sphere, which can be done in a number of different
ways.

The current version of the \texttt{icosa} package uses a single
tessellation method, which requires the least amount of information to
provide a consistent output: The \texttt{"meanGC"} method uses spherical
functions to calculate new points directly on the great circles that
connect points which are on a single edge without any sort of
projection. The internal points are calculated by connecting the newly
formed points on the edges. This results in some scatter for these
internal points, as their position depends on the pair of edges that are
connected. In this method, the points are defined as their centroids
projected to the surface of the sphere, which results in a systematic
increase in cell area as the center of the tessellated face is
approached. Therefore, the answer to the question of the different
tessellation vectors is: the number of faces will be equal as that is
set by the total product of the tessellation vector, but as every
tessellation round includes the above described procedure, the cell
areas, cell shapes will be somewhat different with these. In the future,
multiple tessellation methods are to be incorporated that produce grid
cells with exactly the same areas just to mention one.

As grid complexity increases the time to create the structure increases
as well (The highest resolution grid so far was the \texttt{c(10,10,4)}
trigrid, which took about 2,500 seconds using a single thread of an
Intel Xeon E5-1620 processor, it had 3,200,000 faces, the mean edge
length of 0.17 degrees (20km) and its size was almost 2GB). Performance
also becomes an issue with very large tessellation values, as they
currently incorporate distance matrix calculations (will be updated
later, if required).

A rectangular grid has an additional problem that is not solved by
triangular replacement, which is the definition of neighbouring cells.
With both the rectangular and the triangular grid, two types of possible
connections exist: cells can share either one or two vertices (an edge),
which leads to problems with cell to cell relationship calculations. The
inversion of the triangular grid solves this problem: if every center of
the face becomes a new vertex a hexagonal pattern emerges, which creates
a neighbourhood pattern where the neighbouring faces can share exactly
two vertices only. Every resolution triangular grid can be turned to a
penta-hexagonal one, which is directly created by the
\texttt{hexagrid()} function.

\begin{Shaded}
\begin{Highlighting}[]
\CommentTok{# create a hexagrid object}
\NormalTok{hLow <-}\StringTok{ }\KeywordTok{hexagrid}\NormalTok{()}

\CommentTok{# plot it in 3d}
\KeywordTok{plot3d}\NormalTok{(hLow, }\DataTypeTok{guides=}\NormalTok{F)}
\end{Highlighting}
\end{Shaded}

\includegraphics{icosa_files/figure-latex/hexagrid1-1.png}

By default (\texttt{tessellation=1}), the \texttt{hexagrid()} function
inverts the regular icosahedron, creating a regular
pentagonal-dodecahedron. This object paradoxically has no hexagonal
faces. Increasing the tessellation vector, however, will add these,
while keeping the 12 pentagonal faces at the positions which were
originally containing the icosahedron's vertices.

\begin{Shaded}
\begin{Highlighting}[]
\CommentTok{# create a hexagrid object}
\NormalTok{hLow <-}\StringTok{ }\KeywordTok{hexagrid}\NormalTok{(}\KeywordTok{c}\NormalTok{(}\DecValTok{4}\NormalTok{,}\DecValTok{4}\NormalTok{))}

\CommentTok{# plot it in 3d}
\KeywordTok{plot3d}\NormalTok{(hLow)}
\end{Highlighting}
\end{Shaded}

\includegraphics{icosa_files/figure-latex/hexagrid-1.png}

The function of the tessellation vector is exactly the same as for the
\texttt{trigrid()} function, which is invoked by the \texttt{hexagrid()}
function before the inversion is implemented. This naturally leads to an
equality between the vertex numbers of the hexagrid and face numbers of
the trigrid, and the face numbers of the trigrid and the vertex numbers
of the hexagrid objects.

All methods that are implemented for the trigrid are implemented for the
hexagrid as well. The examples that follow use the two types of grids at
random, and work interchangably.

\subsection{2.2. Grid structure}\label{grid-structure}

The grids implented by this package represent compound objects that have
different `dimensions'. For example, grids represent both a regular 3d
object structure and an object of interconnected cells. The primary 3d
structure of the grid is similar to a generic 3d .obj file structure.
There are two main tables: one contains the grid vertex coordinates and
the other contains which coordinates form which faces. This information
is stored by the vertices and faces slots, respectively:

\begin{Shaded}
\begin{Highlighting}[]
\CommentTok{# the beginning of the vertices matrix}
\KeywordTok{head}\NormalTok{(gLow}\OperatorTok{@}\NormalTok{vertices)}
\end{Highlighting}
\end{Shaded}

\begin{verbatim}
##            x         y        z
## P1    0.0000    0.0000 6371.007
## P2 -418.9419 -136.1225 6355.760
## P3    0.0000 -440.5015 6355.760
## P4  418.9419 -136.1225 6355.760
## P5  258.9203  356.3732 6355.760
## P6 -258.9203  356.3732 6355.760
\end{verbatim}

\begin{Shaded}
\begin{Highlighting}[]
\CommentTok{# the beginning of the faces matrix}
\KeywordTok{head}\NormalTok{(gLow}\OperatorTok{@}\NormalTok{faces)}
\end{Highlighting}
\end{Shaded}

\begin{verbatim}
##    [,1] [,2] [,3]
## F1 "P1" "P2" "P3"
## F2 "P1" "P3" "P4"
## F3 "P1" "P5" "P4"
## F4 "P1" "P5" "P6"
## F5 "P1" "P2" "P6"
## F6 "P2" "P6" "P7"
\end{verbatim}

The information content is stored and all the calculations are executed
in XYZ Cartesian space instead of a polar coordinate system. This
facilitates the definition of additional projection methods, potential
grid-grid interaction, 3d plotting and calculations, and it also permits
higher overall flexibility. The Cartesian coordinates are based on the
value of the grid radius and center.

\begin{Shaded}
\begin{Highlighting}[]
\CommentTok{# grid radius}
\NormalTok{gLow}\OperatorTok{@}\NormalTok{r}
\end{Highlighting}
\end{Shaded}

\begin{verbatim}
## [1] 6371.007
\end{verbatim}

\begin{Shaded}
\begin{Highlighting}[]
\CommentTok{# grid center}
\NormalTok{gLow}\OperatorTok{@}\NormalTok{center}
\end{Highlighting}
\end{Shaded}

\begin{verbatim}
## [1] 0 0 0
\end{verbatim}

The centers of the faces can also be directly accessed in a format that
is similar to the grid vertices format:

\begin{Shaded}
\begin{Highlighting}[]
\KeywordTok{head}\NormalTok{(gLow}\OperatorTok{@}\NormalTok{faceCenters)}
\end{Highlighting}
\end{Shaded}

\begin{verbatim}
##            x         y        z
## F1 -139.7730 -192.3810 6366.568
## F2  139.7730 -192.3810 6366.568
## F3  226.1574   73.4830 6366.568
## F4    0.0000  237.7960 6366.568
## F5 -226.1574   73.4830 6366.568
## F6 -453.0188  147.1947 6353.176
\end{verbatim}

Both the \texttt{vertices} and the \texttt{faceCenters()} slots are
accessible using the shorthand functions \texttt{vertices()} and
\texttt{centers()}, which also do coordinate transformations, if
requested.

The vertices forming the edges (these are not ordered in the current
version) can be extracted from the \texttt{edges} slot:

\begin{Shaded}
\begin{Highlighting}[]
\KeywordTok{head}\NormalTok{(gLow}\OperatorTok{@}\NormalTok{edges)}
\end{Highlighting}
\end{Shaded}

\begin{verbatim}
##    [,1]  [,2] 
## E1 "P1"  "P3" 
## E2 "P1"  "P4" 
## E3 "P3"  "P4" 
## E4 "P39" "P22"
## E5 "P39" "P40"
## E6 "P22" "P40"
\end{verbatim}

Each grid has an orientation which is stored in the \texttt{orientation}
slot. The values are in radians, and denote the xyz rotation relative to
the default. The faces and vertices table are organized so that both
vertices and faces spiral down from the zenith point to the nadir. This
can be visualized in 3d using the \texttt{gridlabs3d()} function.

\begin{Shaded}
\begin{Highlighting}[]
\KeywordTok{plot3d}\NormalTok{(gLow)}
\KeywordTok{gridlabs3d}\NormalTok{(gLow, }\DataTypeTok{type=}\StringTok{"v"}\NormalTok{, }\DataTypeTok{col=}\StringTok{"blue"}\NormalTok{, }\DataTypeTok{cex=}\FloatTok{0.6}\NormalTok{)}
\end{Highlighting}
\end{Shaded}

\includegraphics{icosa_files/figure-latex/tri6-1.png}

The grid orientation can be changed using the \texttt{rotate()}
function. To see the effect of this on the 3d plots, compare the
orientations of the grids using the \texttt{guides3d()} function that
displays the polar gridding oriented to match the cartesian coordinate
system.

\begin{Shaded}
\begin{Highlighting}[]
\NormalTok{gLow2 <-}\StringTok{ }\KeywordTok{rotate}\NormalTok{(gLow) }\CommentTok{# random rotation}
\KeywordTok{plot3d}\NormalTok{(gLow2)}
\KeywordTok{guides3d}\NormalTok{(}\DataTypeTok{col=}\StringTok{"green"}\NormalTok{)}
\end{Highlighting}
\end{Shaded}

\includegraphics{icosa_files/figure-latex/tri8-1.png}

\subsection{2.3. Plotting}\label{plotting}

\subsubsection{2.3.1. Three-dimensional
plots}\label{three-dimensional-plots}

Both 3d and 2d plotting are incorporated in the package. As the grid
structure exists in 3d space, 3d is the default plotting scheme which is
implemented with the package \texttt{rgl}'. All 3d plotting functions
pass arguments to either the \texttt{points3d()}, \texttt{segments3d()},
\texttt{triangles3d()} and \texttt{text3d()} functions.

The \texttt{plot3d()} method of the grids call for either the border
plotting function \texttt{lines3d()} or the face plotting function
\texttt{faces3d()}. In a workflow involving 3d plotting, these functions
are used usually to create a compound plot representing different types
of information. Experiment with these to optimize the 3d plotting
experience.

The inner sphere is plotted by default, but can be turned off by setting
the \texttt{sphere} argument of the \texttt{plot3d()} function to
\texttt{FALSE}. The radius of the sphere can also be set using this
argument. In case it is not set by the user, it defaults to the distance
of the planar face center from the center of the grid.

The 3d plots so far showed only linear edges, but the plotting of arcs
can be forced by setting the \texttt{arcs} argument to \texttt{TRUE}.

\begin{Shaded}
\begin{Highlighting}[]
\KeywordTok{plot3d}\NormalTok{(tri, }\DataTypeTok{guides=}\NormalTok{F, }\DataTypeTok{arcs=}\NormalTok{T, }\DataTypeTok{sphere=}\DecValTok{6300}\NormalTok{)}
\end{Highlighting}
\end{Shaded}

\includegraphics{icosa_files/figure-latex/first2-1.png}

\subsubsection{2.3.2. Two-dimensional
plots}\label{two-dimensional-plots}

The nature of the triangular/hexagonal grids is that they are intuitive
in 3 dimensions, but behave cumbersome in 2d projections. Still, in any
sort of printed or software publications, maps are the primary way to
publish geographic data, which renders the projections very important.
This part of the package is linked to the \texttt{sp} packages, which
deal with the projection of data.

Each grid can be converted to either a \texttt{SpatialLines} or a
\texttt{SpatialPolygons} object defined by the \texttt{sp} package. Two
dimensional plotting can only happen if the 2d representation is
calculated, which is (to save computation time) not automatic, but can
be called for on demand.

\begin{Shaded}
\begin{Highlighting}[]
\NormalTok{hLow <-}\StringTok{ }\KeywordTok{hexagrid}\NormalTok{(}\KeywordTok{c}\NormalTok{(}\DecValTok{4}\NormalTok{,}\DecValTok{4}\NormalTok{), }\DataTypeTok{sp=}\OtherTok{TRUE}\NormalTok{)}
\CommentTok{# After this procedure finishes, a regular 2d plotting function can be invoked:}
\KeywordTok{plot}\NormalTok{(hLow)}
\end{Highlighting}
\end{Shaded}

\includegraphics{icosa_files/figure-latex/tri10-1.png}

Here are some additional examples of projections using the World Borders
Dataset (2) that can be accessed in the
\texttt{SpatialPolygonsDataframe} format using the following chunk of
code:

Here are some additional examples of projections using the `z1'
resolution of landy polygons from the OSM archive (2) that can be
accessed in the \texttt{SpatialPolygons} format using the following
chunk of code.

\begin{Shaded}
\begin{Highlighting}[]
\CommentTok{# use the rgdal package}
\KeywordTok{library}\NormalTok{(rgdal)}

\CommentTok{# file path}
\NormalTok{file <-}\StringTok{ }\KeywordTok{system.file}\NormalTok{(}\StringTok{"extdata"}\NormalTok{, }\StringTok{"land_polygons_z1.shx"}\NormalTok{, }\DataTypeTok{package =} \StringTok{"icosa"}\NormalTok{)}

\CommentTok{# read in the shape file}
\NormalTok{wo <-}\StringTok{ }\KeywordTok{readOGR}\NormalTok{(file, }\StringTok{"land_polygons_z1"}\NormalTok{)}
\end{Highlighting}
\end{Shaded}

A grid can be plotted easilly with this map, after their projection
methods are adjusted:

\begin{Shaded}
\begin{Highlighting}[]
\CommentTok{# transform the land data to long-lat coordinates}
\NormalTok{wo <-}\StringTok{ }\KeywordTok{spTransform}\NormalTok{(wo, gLow}\OperatorTok{@}\NormalTok{proj4string)}

\CommentTok{#triangular grid}
\NormalTok{gLow<-}\KeywordTok{newsp}\NormalTok{(gLow)}

\CommentTok{# # load in a map}
\CommentTok{# plot the grid (default longitude/latitude)}
\KeywordTok{plot}\NormalTok{(gLow, }\DataTypeTok{border=}\StringTok{"gray"}\NormalTok{, }\DataTypeTok{lty=}\DecValTok{1}\NormalTok{)}
 
\CommentTok{# the reconstruction}
\KeywordTok{lines}\NormalTok{(wo, }\DataTypeTok{lwd=}\DecValTok{2}\NormalTok{, }\DataTypeTok{col=}\StringTok{"blue"}\NormalTok{)}
\end{Highlighting}
\end{Shaded}

\includegraphics{icosa_files/figure-latex/examplePaleo-1.png}

The \texttt{gridlabs()} function can also be of use here to locate the
vertices and faces of the plotted grid. The \texttt{type} argument is
used to choose which part of the grid is to be shown. The rest of the
argumnets are passed to the \texttt{text()} function.

\begin{Shaded}
\begin{Highlighting}[]
\CommentTok{# a very low resolution hexagrid}
\NormalTok{hVeryLow<-}\KeywordTok{hexagrid}\NormalTok{(}\KeywordTok{c}\NormalTok{(}\DecValTok{4}\NormalTok{))}
\CommentTok{# add 2d component}
\NormalTok{hVeryLow<-}\KeywordTok{newsp}\NormalTok{(hVeryLow)}
\CommentTok{# the Robinson projection}
\NormalTok{robin <-}\StringTok{ }\KeywordTok{CRS}\NormalTok{(}\StringTok{"+proj=robin"}\NormalTok{)}
\CommentTok{# plot with labels}
\KeywordTok{plot}\NormalTok{(hVeryLow, }\DataTypeTok{projargs=}\NormalTok{robin)}
\KeywordTok{gridlabs}\NormalTok{(hVeryLow, }\DataTypeTok{type=}\StringTok{"f"}\NormalTok{, }\DataTypeTok{cex=}\FloatTok{0.6}\NormalTok{,}\DataTypeTok{projargs=}\NormalTok{robin)}
\end{Highlighting}
\end{Shaded}

\includegraphics{icosa_files/figure-latex/tri12-1.png}

Similarly useful can be the \texttt{pos()} function, that retrieves the
position of a named element in the grid, e.g.~vertices and face centers:

\begin{Shaded}
\begin{Highlighting}[]
\KeywordTok{pos}\NormalTok{(hLow, }\KeywordTok{c}\NormalTok{(}\StringTok{"P2"}\NormalTok{, }\StringTok{"F12"}\NormalTok{, }\OtherTok{NA}\NormalTok{))}
\end{Highlighting}
\end{Shaded}

\begin{verbatim}
##      long      lat
## P2    -54 87.86095
## F12   -18 82.07063
## <NA>   NA       NA
\end{verbatim}

\subsection{2.4. Layers}\label{layers}

The grid itself operates as a scaffold for all kinds operations we can
do based on data which can be organized in layers. At this moment, the
layers are built on vectors, but in the next major update of the package
they will incorporate both memory and harddrive-stored data similar to
the \texttt{RasterLayer} class objects defined in the \texttt{raster}
package.

Currently only the \texttt{facelayer} class is defined, which link
individual values to the faces of a \texttt{trigrid} or
\texttt{hexagrid} class object.

\begin{Shaded}
\begin{Highlighting}[]
\NormalTok{fl1<-}\KeywordTok{facelayer}\NormalTok{(gLow) }\CommentTok{# the argument is the grid object to which the layer is linked}
\NormalTok{fl1}
\end{Highlighting}
\end{Shaded}

\begin{verbatim}
## class        : facelayer
## linked grid  : 'gLow' (name), trigrid (class), 4,4 (tessellation)
## dimensions   : 5120 (values) @ mean edge length: 481.07 km, 4.33 degrees
## values       :  logical 
## max value    :  NA 
## min value    :  NA 
## missing      :  5120
\end{verbatim}

\begin{Shaded}
\begin{Highlighting}[]
\KeywordTok{str}\NormalTok{(fl1)}
\end{Highlighting}
\end{Shaded}

\begin{verbatim}
## Formal class 'facelayer' [package "icosa"] with 6 slots
##   ..@ grid        : chr "gLow"
##   ..@ tessellation: num [1:2] 4 4
##   ..@ gridclass   : chr "trigrid"
##   .. ..- attr(*, "package")= chr "icosa"
##   ..@ names       : chr [1:5120] "F1" "F2" "F3" "F4" ...
##   ..@ values      : logi [1:5120] NA NA NA NA NA NA ...
##   ..@ length      : int 5120
\end{verbatim}

The \texttt{facelayer} has the same number of values as the the number
of faces in the linked grid, accessed by the \texttt{length()} function

\begin{Shaded}
\begin{Highlighting}[]
\KeywordTok{length}\NormalTok{(fl1)}
\end{Highlighting}
\end{Shaded}

\begin{verbatim}
## [1] 5120
\end{verbatim}

The stored values can be assigned or shown by the values function:

\begin{Shaded}
\begin{Highlighting}[]
\KeywordTok{values}\NormalTok{(fl1) <-}\DecValTok{1}\OperatorTok{:}\KeywordTok{length}\NormalTok{(fl1)}
\KeywordTok{values}\NormalTok{(fl1)[}\DecValTok{1}\OperatorTok{:}\DecValTok{10}\NormalTok{]}
\end{Highlighting}
\end{Shaded}

\begin{verbatim}
##  [1]  1  2  3  4  5  6  7  8  9 10
\end{verbatim}

Besides storage and data manipulation, layers can be especially useful
for plotting data. For logical data the \texttt{faces3d()} function will
indicate which faces are occupied.

\begin{Shaded}
\begin{Highlighting}[]
\NormalTok{a <-}\KeywordTok{facelayer}\NormalTok{(gLow)}
\KeywordTok{values}\NormalTok{(a) <-}\StringTok{ }\KeywordTok{sample}\NormalTok{(}\KeywordTok{c}\NormalTok{(T,F), }\KeywordTok{length}\NormalTok{(a), }\DataTypeTok{replace=}\NormalTok{T)}
\CommentTok{# plot the grid first}
\KeywordTok{plot3d}\NormalTok{(gLow, }\DataTypeTok{guides=}\NormalTok{F)}

\CommentTok{# invoke lower level plotting for the facelayer }
\CommentTok{# (draws on previously plotted rgl environemnts)}
\KeywordTok{faces3d}\NormalTok{(a, }\DataTypeTok{col=}\StringTok{"green"}\NormalTok{)}
\end{Highlighting}
\end{Shaded}

\includegraphics{icosa_files/figure-latex/tri18-1.png}

This is the lower level graphic function, that is called when the
\texttt{plot3d()} method of the \texttt{facelayer} is called. For
numeric data, heatmaps are built automatically based on the range of the
data. Let's examine the basic case, where its number in sequence is
assigned to every face.

The colors of the heatmaps can be changed by adding standard color names
to the \texttt{col} argument:

\begin{Shaded}
\begin{Highlighting}[]
\CommentTok{# new layer}
\CommentTok{# grid frame}
\KeywordTok{plot3d}\NormalTok{(gLow)}
\CommentTok{# the heatmap}
\KeywordTok{faces3d}\NormalTok{(a, }\DataTypeTok{col=}\KeywordTok{c}\NormalTok{(}\StringTok{"green"}\NormalTok{, }\StringTok{"brown"}\NormalTok{)) }
\end{Highlighting}
\end{Shaded}

\includegraphics{icosa_files/figure-latex/tri19b-1.png}

Categorical values can also be stored and plotted with the facelayer. By
default, these values will be plotted with random colours, without a
legend.

\begin{Shaded}
\begin{Highlighting}[]
\CommentTok{# new layer}
\NormalTok{catLayer<-}\KeywordTok{facelayer}\NormalTok{(hLow)}

\CommentTok{# assign random information}
\NormalTok{catLayer}\OperatorTok{@}\NormalTok{values<-}\KeywordTok{sample}\NormalTok{(}\KeywordTok{c}\NormalTok{(}\StringTok{"one"}\NormalTok{,}\StringTok{"two"}\NormalTok{,}\StringTok{"three"}\NormalTok{),}\KeywordTok{length}\NormalTok{(catLayer), }\DataTypeTok{replace=}\NormalTok{T)}

\KeywordTok{plot}\NormalTok{(catLayer)}
\end{Highlighting}
\end{Shaded}

\includegraphics{icosa_files/figure-latex/categ1-1.png}

\section{3. Application}\label{application}

\subsection{3.1. Lookup}\label{lookup}

Until this point only those features of the package were demonstrated
that have no practicality on their own. All real world application of a
gridding scheme relies on the capacity to look up coordinates and assign
them to grid cells. The overall performance of the package boils down to
the speed of this procedure. \texttt{icosa} uses a very efficient
point-in-tetrhedron check to get the assigned cells to each set of
coordinates. In the case of the \texttt{trigrid}, every face on the
surface of the grid outlines a tetrahedron with the center of the
object. At high resolutions this in itself can be very slow, especially
if the number of queries is large, hence the necessity of the skeleton
slot and the multiple levels of tessellations. With the \texttt{meanGC}
tessellation method, the vertices of the input do not change, which
means that every level of resolution can be retained when multiple
rounds of tessellation happen. This allows the implementation of a
hierarchical lookup algorithm, which searches the position of a point
given by progressively refining the resolution, so an exhaustive lookup
is not required.

\subsubsection{\texorpdfstring{3.1.1. The `locate()' function - point
query}{3.1.1. The locate() function - point query}}\label{the-locate-function---point-query}

The most straightforward implementation is the \texttt{locate()}
function which is used to find the position of a set of points on the
grid:

\begin{Shaded}
\begin{Highlighting}[]
\CommentTok{# generate 5000 random coordinates on a sphere of default radius}
\NormalTok{pointdat <-}\StringTok{ }\KeywordTok{rpsphere}\NormalTok{(}\DecValTok{5000}\NormalTok{)}

\CommentTok{# and locate them on the grid 'gLow'}
\NormalTok{cells<-}\KeywordTok{locate}\NormalTok{(gLow, pointdat)}

\CommentTok{# the return of this function is vector of cell names}
\KeywordTok{head}\NormalTok{(cells)}
\end{Highlighting}
\end{Shaded}

\begin{verbatim}
## [1] "F4429" "F4111" "F678"  "F4708" "F1853" "F4911"
\end{verbatim}

The function accepts matrices in longitude-latitude, and XYZ format as
well. An object of the \texttt{SpatialPoints} class defined in the
package \texttt{sp} can alse be provided as input. In the case of the
polar coordinate entry, the coordinates will be transformed to the xyz
Cartesian coordinate system using the default radius. This function
returns the names of the faces that the points fell on. In the case of
points that fall on vertices or edges (which is extremely unlikely with
real world data), the returned values are by default NAs. The
\texttt{locate()} function is especially powerful if it combined with
the\texttt{table()} and \texttt{tapply()} functions or similar types of
iterators:

\begin{Shaded}
\begin{Highlighting}[]
\NormalTok{tCell <-}\StringTok{ }\KeywordTok{table}\NormalTok{(cells)}
\NormalTok{fl <-}\StringTok{ }\KeywordTok{facelayer}\NormalTok{(gLow,}\DecValTok{0}\NormalTok{)}
\CommentTok{# [] invokes a method that save the values to places that }
\CommentTok{# correspond to the names attribute of tCell}
\NormalTok{fl[] <-tCell }\CommentTok{#}
\CommentTok{# heat map of the point densities}
\KeywordTok{plot3d}\NormalTok{(fl)}
\end{Highlighting}
\end{Shaded}

\includegraphics{icosa_files/figure-latex/tri20b-1.png}

This function operates just as fine with the \texttt{hexagrid} object,
and uses subfaces to locate the points. Every hexagonal face consists of
6 subfaces and every pentagonal face contains 5 subfaces.

The performance of the \texttt{locate()} function is linearly related to
the number of queries. It is also positively related to the grid
resolution, although larger tessellation values will increase
computation time more than using multiple levels of tessellations.

\subsubsection{3.1.2. The occupied()
function}\label{the-occupied-function}

For presence-absence values the function \texttt{occupied()} can be
used. It returns a \texttt{facelayer} class objecte with logical values
(\texttt{TRUE} when the face is occupied an \texttt{FALSE} when the face
is not).

The example below shows how the occupied cells can be shown with the
points:

\begin{Shaded}
\begin{Highlighting}[]
\CommentTok{# run function only on the first 300}
\NormalTok{fl<-}\KeywordTok{occupied}\NormalTok{(hLow, pointdat[}\DecValTok{1}\OperatorTok{:}\DecValTok{300}\NormalTok{,])}

\CommentTok{# the plot function can also be applied to the facelayer object}
\KeywordTok{plot}\NormalTok{(fl, }\DataTypeTok{col=}\StringTok{"blue"}\NormalTok{)}

\CommentTok{# show the points as well}
\KeywordTok{points}\NormalTok{(}\KeywordTok{CarToPol}\NormalTok{(pointdat[}\DecValTok{1}\OperatorTok{:}\DecValTok{300}\NormalTok{,]), }\DataTypeTok{col=}\StringTok{"red"}\NormalTok{, }\DataTypeTok{pch=}\DecValTok{3}\NormalTok{, }\DataTypeTok{cex=}\FloatTok{0.7}\NormalTok{)}
\end{Highlighting}
\end{Shaded}

\includegraphics{icosa_files/figure-latex/tri22-1.png}

Naturally the grid can be shown as well, for instance with
\texttt{lines()}:

\begin{Shaded}
\begin{Highlighting}[]
\CommentTok{# the plot function can also be applied to the facelayer object}
\KeywordTok{plot}\NormalTok{(fl, }\DataTypeTok{col=}\StringTok{"blue"}\NormalTok{)}

\KeywordTok{points}\NormalTok{(}\KeywordTok{CarToPol}\NormalTok{(pointdat[}\DecValTok{1}\OperatorTok{:}\DecValTok{300}\NormalTok{,]), }\DataTypeTok{col=}\StringTok{"red"}\NormalTok{, }\DataTypeTok{pch=}\DecValTok{3}\NormalTok{, }\DataTypeTok{cex=}\FloatTok{0.7}\NormalTok{)}
\KeywordTok{lines}\NormalTok{(hLow, }\DataTypeTok{col=}\StringTok{"gray"}\NormalTok{)}
\end{Highlighting}
\end{Shaded}

\includegraphics{icosa_files/figure-latex/tri22b-1.png}

The \texttt{occupied()} function also applies to various other object
types and behaves as a wrapper function around methods that return which
faces are occupied by the input objects. Most notable among these is the
\texttt{SpatialPolygons}, \texttt{SpatialLines}, and
\texttt{SpatialPoints} classes defined by the package \texttt{sp}. The
method changes the coordinate reference system (CRS) of the input object
is used to transform it to the spherical model first, and then the
function transforms the coordinates to XYZ Cartesian space.

\subsubsection{3.1.3. Handling raster-type
data}\label{handling-raster-type-data}

Most global data compilations use raster formats to store information.
These data can be fitted to the icosahedral grids using the
\texttt{resample()} function. The arguments of this function depend on
the nature and interpretation of the data points. As resampling requires
some form of interpolation, it needs assumptions on the representativity
of the measurements. Each original data point can be thought of either
as an entity that represent the entire cell or only the center of the
cell. In the first case the original raster object needs to be upscaled
with the nearest neighbour method, and in the latter, another form of
interpolation is necessary (e.g.~the bilinear or bicubic resampling).
The `method' argument of this function is passed to the
\texttt{resample()} function in the \texttt{raster} package, and is used
to generate higher resolution data from the original raster.The
\texttt{resample()} function can also be used to upscale, or downscale a
\texttt{facelayer} linked to \texttt{trigrid} or \texttt{hexagrid}
object as well.

\subsection{3.2. Surface-graph
representation}\label{surface-graph-representation}

The grid structure is a compound object, can also be understood as a
graph of connected faces. This representation is efficiently implemented
using the \texttt{igraph} package. On default, \texttt{igraph}
represenation of the grid is added to the \texttt{graph} slot of the
grid object. In this graph, each face is connected to its direct
neighbours, which allows etheir efficient lookup, the implementation of
shortest path algorithms and more.

\subsubsection{3.2.1. Neighbours}\label{neighbours}

The most direct application of this representation is the
\texttt{vicinity()} function that allows the user to look up cells that
are closest to a focal cell, without calculating distance matrices. This
particular example gets all the neighbouring cells of the \texttt{F125}
cell.

\begin{Shaded}
\begin{Highlighting}[]
\CommentTok{# calculate a very coarse resolution grid}
\NormalTok{gVeryLow<-}\KeywordTok{trigrid}\NormalTok{(}\DecValTok{8}\NormalTok{, }\DataTypeTok{sp=}\NormalTok{T)}
\CommentTok{# names of faces that are neighbours to face F125}
\NormalTok{facenames<-}\KeywordTok{vicinity}\NormalTok{(gVeryLow, }\StringTok{"F125"}\NormalTok{)}
\CommentTok{# plot a portion of the grid}
\KeywordTok{plot}\NormalTok{(gVeryLow, }\DataTypeTok{xlim=}\KeywordTok{c}\NormalTok{(}\DecValTok{0}\NormalTok{,}\DecValTok{180}\NormalTok{), }\DataTypeTok{ylim=}\KeywordTok{c}\NormalTok{(}\DecValTok{0}\NormalTok{,}\DecValTok{90}\NormalTok{))}
\CommentTok{# plot the original and the neighbouring faces}
\KeywordTok{plot}\NormalTok{(gVeryLow}\OperatorTok{@}\NormalTok{sp[facenames], }\DataTypeTok{col=}\StringTok{"red"}\NormalTok{, }\DataTypeTok{add=}\NormalTok{T)}
\CommentTok{# the names of all the cells}
\KeywordTok{gridlabs}\NormalTok{(gVeryLow, }\DataTypeTok{type=}\StringTok{"f"}\NormalTok{, }\DataTypeTok{cex=}\FloatTok{0.5}\NormalTok{)}
\end{Highlighting}
\end{Shaded}

\includegraphics{icosa_files/figure-latex/vic1-1.png}

\subsubsection{\texorpdfstring{3.2.2. Using `igraph' in geographic
calculations}{3.2.2. Using igraph in geographic calculations}}\label{using-igraph-in-geographic-calculations}

Using a separate \texttt{igraph} class object can be especially useful
when subsets of the grids are to be used for an analysis or simulation.

\begin{Shaded}
\begin{Highlighting}[]
\CommentTok{# attach igraph}
\KeywordTok{library}\NormalTok{(igraph)}
\end{Highlighting}
\end{Shaded}

\begin{verbatim}
## 
## Attaching package: 'igraph'
\end{verbatim}

\begin{verbatim}
## The following objects are masked from 'package:icosa':
## 
##     edges, vertices
\end{verbatim}

\begin{verbatim}
## The following object is masked from 'package:raster':
## 
##     union
\end{verbatim}

\begin{verbatim}
## The following objects are masked from 'package:stats':
## 
##     decompose, spectrum
\end{verbatim}

\begin{verbatim}
## The following object is masked from 'package:base':
## 
##     union
\end{verbatim}

Please note that \texttt{igraph} masks out some of the auxilliary
functions written in this package as well. Naturally, you can use the
\texttt{induced\_subgraph()} function of the \texttt{igraph} package
directly on the grid representation of the grid.

\begin{Shaded}
\begin{Highlighting}[]
\NormalTok{faces<-}\KeywordTok{paste}\NormalTok{(}\StringTok{"F"}\NormalTok{, }\DecValTok{1}\OperatorTok{:}\DecValTok{10}\NormalTok{, }\DataTypeTok{sep=}\StringTok{""}\NormalTok{)}
\NormalTok{subGraph <-}\StringTok{ }\KeywordTok{induced_subgraph}\NormalTok{(gVeryLow}\OperatorTok{@}\NormalTok{graph,faces) }
\KeywordTok{plot}\NormalTok{(subGraph)}
\end{Highlighting}
\end{Shaded}

\includegraphics{icosa_files/figure-latex/ggraph4-1.png}

The subsetting of the grid will also subset the \texttt{igraph} class
representation:

\begin{Shaded}
\begin{Highlighting}[]
\NormalTok{lowGraph<-gLow[}\DecValTok{1}\OperatorTok{:}\DecValTok{12}\NormalTok{]}\OperatorTok{@}\NormalTok{graph}
\end{Highlighting}
\end{Shaded}

or you can create it from a logical \texttt{facelayer}, for example from
the occupied cells of the land data we imported earlier:

\begin{Shaded}
\begin{Highlighting}[]
\CommentTok{# look up the polygons}
\NormalTok{landFaces<-}\KeywordTok{occupied}\NormalTok{(hLow, wo)}

\CommentTok{# create a new grid from a facelayer}
\NormalTok{landGraph<-}\KeywordTok{gridgraph}\NormalTok{(landFaces)}
\KeywordTok{plot}\NormalTok{(landFaces, }\DataTypeTok{col=}\StringTok{"brown"}\NormalTok{)}
\end{Highlighting}
\end{Shaded}

\includegraphics{icosa_files/figure-latex/ggraph4b-1.png}

This particular graph is a rough estimate for the presence of
terrestrial settings, and can be useful for path calculations.

\begin{Shaded}
\begin{Highlighting}[]
\CommentTok{# shortest path in igraph}
\NormalTok{path <-}\StringTok{ }\KeywordTok{shortest_paths}\NormalTok{(landGraph, }\DataTypeTok{from=}\StringTok{"F432"}\NormalTok{, }\DataTypeTok{to=}\StringTok{"F1073"}\NormalTok{, }\DataTypeTok{output=}\StringTok{"vpath"}\NormalTok{)}
\CommentTok{# the names of the cells in order}
\NormalTok{cells<-path}\OperatorTok{$}\NormalTok{vpath[[}\DecValTok{1}\NormalTok{]]}\OperatorTok{$}\NormalTok{name}
\CommentTok{# plot the map}
\KeywordTok{plot}\NormalTok{(landFaces, }\DataTypeTok{col=}\StringTok{"brown"}\NormalTok{, }\DataTypeTok{xlim=}\KeywordTok{c}\NormalTok{(}\DecValTok{0}\NormalTok{,}\DecValTok{90}\NormalTok{), }\DataTypeTok{ylim=}\KeywordTok{c}\NormalTok{(}\DecValTok{0}\NormalTok{,}\DecValTok{90}\NormalTok{))}
\CommentTok{# make a subset of the grid - which corresponds to the path}
\NormalTok{routeGrid<-hLow[cells]}
\CommentTok{# plot the path}
\KeywordTok{plot}\NormalTok{(routeGrid, }\DataTypeTok{col=}\StringTok{"red"}\NormalTok{, }\DataTypeTok{add=}\NormalTok{T)}
\end{Highlighting}
\end{Shaded}

\includegraphics{icosa_files/figure-latex/ggraph6-1.png}

The shortest path using grid cells is a suboptimal estimate of the
actual shortest route between two points, as the graph structure limits
the angles the path can turn to. A future update will include a function
that allows more accurate estimates of the actual shortest paths.

Random walk simulations can also be built using the graph represetation.
In this example a random walker will walk 100 steps on the grid,
starting from face \texttt{F432}.

\begin{Shaded}
\begin{Highlighting}[]
\CommentTok{# plot the map}
\KeywordTok{plot}\NormalTok{(landFaces, }\DataTypeTok{col=}\StringTok{"brown"}\NormalTok{, }\DataTypeTok{xlim=}\KeywordTok{c}\NormalTok{(}\DecValTok{0}\NormalTok{,}\DecValTok{90}\NormalTok{), }\DataTypeTok{ylim=}\KeywordTok{c}\NormalTok{(}\DecValTok{0}\NormalTok{,}\DecValTok{90}\NormalTok{))}
\CommentTok{# create a random walk from source cell with a given no. of steps}
\NormalTok{randomWalk <-}\StringTok{ }\KeywordTok{random_walk}\NormalTok{(landGraph, }\DataTypeTok{steps=}\DecValTok{100}\NormalTok{, }\DataTypeTok{start=}\StringTok{"F432"}\NormalTok{)}
\CommentTok{# the names of the cells visited by the random walker}
\NormalTok{cells<-randomWalk}\OperatorTok{$}\NormalTok{name}
\CommentTok{# the source cell}
\KeywordTok{plot}\NormalTok{(hLow[}\StringTok{"F432"}\NormalTok{], }\DataTypeTok{col=}\StringTok{"green"}\NormalTok{,}\DataTypeTok{add=}\NormalTok{T)}
\CommentTok{# the centers of these faces}
\NormalTok{centers<-}\KeywordTok{CarToPol}\NormalTok{(hLow}\OperatorTok{@}\NormalTok{faceCenters[cells,], }\DataTypeTok{norad=}\NormalTok{T)}
\CommentTok{# draw the lines of the random walk}
\ControlFlowTok{for}\NormalTok{(i }\ControlFlowTok{in} \DecValTok{2}\OperatorTok{:}\KeywordTok{nrow}\NormalTok{(centers))\{}
    \KeywordTok{segments}\NormalTok{(}\DataTypeTok{x0=}\NormalTok{centers[i}\DecValTok{-1}\NormalTok{,}\DecValTok{1}\NormalTok{], }\DataTypeTok{y0=}\NormalTok{centers[i}\DecValTok{-1}\NormalTok{,}\DecValTok{2}\NormalTok{], }\DataTypeTok{x1=}\NormalTok{centers[i,}\DecValTok{1}\NormalTok{], }\DataTypeTok{y1=}\NormalTok{centers[i,}\DecValTok{2}\NormalTok{], }\DataTypeTok{lwd=}\DecValTok{2}\NormalTok{)}
\NormalTok{\}}
\end{Highlighting}
\end{Shaded}

\includegraphics{icosa_files/figure-latex/ggraph6b-1.png}

\section{Acknowledgements}\label{acknowledgements}

The `icosa' package development is part of a Deutsche
Forschungsgemeinschaft project for global biogeographic analyses (KO
5382/1-1 and KO 5382/1-2) and the Research Unit TERSANE (FOR 2332).
Special thanks are due to all early testers of the project in particular
to: Wolfgang Kiessling, Kilian Eichenseer, Carl Reddin, Vanessa Roden,
Emilia Jarochowska and Andreas Lauchstedt

\section{References}\label{references}

\begin{enumerate}
\def\labelenumi{(\arabic{enumi})}
\tightlist
\item
  Moritz, H. 2000. Geodetic Reference System 1980. Journal of Geodesy,
  74, 128-162.
\item
  \url{http://openstreetmapdata.com/}
\item
  \url{http://www.worldclim.org/}
\end{enumerate}

\end{document}
